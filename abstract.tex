% Francais

Dans le cadre de la démonstration des capacités technologiques actuelles en robotique collaborative, 
ce projet a pour but de mettre en fonctionnement un petit système robotisé de gravure laser sécurisé 
et utilisable par le grand public. L'objectif principal est de présenter le projet à différentes occasions 
afin que le public puisse interagir et choisir la gravure à réaliser sur de petites pièces en bois. Ce projet 
présente plusieurs défis importants, tels que la reconnaissance et le repérage d'objets dans l'espace
par vision, la mise en service générale d'un bras robot collaboratif, la synchronisation laser-robot, 
la sécurité des utilisateurs, le stockage et la distribution des produits finis, ainsi que la programmation
de la graveuse laser et du robot par un système innovant développé par l'entreprise partenaire.

L'ensemble du travail permet de mettre en application une grande variété de connaissances acquises lors
du cursus, notamment la programmation, l’électronique, la communication entre les éléments, la création
d'interfaces graphiques, la robotique et la vision industrielle.




